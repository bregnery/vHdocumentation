\documentclass[12pt]{article}
 
%\usepackage{fullpage}
\usepackage{hyperref}
\usepackage{color}
\usepackage{colortbl}
\usepackage{graphicx}
\DeclareGraphicsExtensions{.pdf,.png,.gif,.jpg}
%\usepackage{tabular}
  
\title{A Study to Enhance the Sensitivity for the Discovery of the Higgs Boson Coupling to Dimuons}
\author{Brendan Regnery, Darin Acosta, Justin Hugon}
\date{\today}
\begin{document}
\maketitle
 
\section{Introduction}

On July 4, 2012, the Higgs boson was discovered by the Compact Muon Solenoid (CMS) and ATLAS experiments 
at the Large Hadron Collider (LHC) in Geneva, Switzerland \cite{HiggsDisc}. The University of Florida is
a participant in the CMS experiment and helped analyze data that led to this discovery.
The Higgs boson is a particle of the Higgs field, which gives mass to matter. Without the Higgs boson, 
the Standard Model (a model of the elementary particles) cannot explain why particles have mass. 
The Higgs boson was first discovered coupling to massive particles (e.g. the W and Z bosons), 
but the current theories predict that, in rare cases, the Higgs should also couple to less massive particles such as the muon. 
Further analysis of these rare channels is needed to confirm the properties of the Higgs boson. 

The smallest Higgs coupling directly observable at the LHC is the Higgs coupling to two muons ($H\rightarrow \mu \mu$). 
A search for $H\rightarrow \mu \mu$ has already been run on CMS data received from the 7 and 8 TeV runs of the LHC. 
This baseline analysis reported an expected limit of $5.1^{+2.3}_{-1.5}$ times the standard model
and an observed limit of 7.4 times the standard model \cite{hmumuPap}. It is projected that, 
at 14 TeV (using the process described in \cite{snow}), standard model 
sensitivity should be obtained with a lumonosity of $~175^{+150}_{-75}$ fb$^{-1}$ \cite{hmumuPap}. 
We hope to improve upon this for the next run of the LHC at 13 TeV \cite{AN2012_459}. 

The baseline analysis optimized three signal categories: a gluon fusion (GF) production mode category, 
a vector boson fusion (VBF) production mode category, and a 2 jet loose category. 
In the associated production of a Higgs boson with a vector boson (VH) production mode, 
the vector boson normally decays to two jets ($V\rightarrow jj$). Therefore, in this analysis, events from the VH production mode
were included in the 2 jet loose category. In order to improve upon the baseline analysis, we created and optimized 
a category for VH (where $H\rightarrow \mu \mu$ and $V\rightarrow jj$) for 8 TeV data.

\section{Methods}

\subsection{Data Samples}

Our first analysis uses 8 TeV simulated data. These are the methods that we used to obtain our simulated data.

\subsubsection{Simulated Signal Samples}

Our GF and VBF signal samples were produced using the \textsc{powheg-box} NLO generator~\cite{powheg1,powheg2,powheg3} 
interfaced with \textsc{pythia} 6~\cite{pythia} for parton showering.
For the associated production samples, we used the \textsc{herwig}++ event generator and 
parton shower program~\cite{herwigpp}.  NLO corrections were calculated using the 
POWHEG implementation built into \textsc{herwig}++. 
In our associated production samples, the W or Z bosons were allowed to decay without restriction.
Each signal sample contains 100,000 events.

For gluon-fusion, VBF, and associated production, we used samples with the mass of the Higgs boson, $m_H$, equal to 125~$\textrm{GeV}/\textrm{c}^{2}$. 
The names of the centrally produced datasets we used are shown in Table \ref{tab:sigDatasets}

\begin{table}[htb]
\caption{Simulated signal dataset names used in the analysis.
\label{tab:sigDatasets}
}
\small
\begin{center}
\begin{tabular}{ |p{12cm}|}
\hline
Signal Dataset Name \\
\hline
\hline
/GluGlu\_HToMM\_M-125\_TuneZ2star\_8TeV-powheg-pythia6/Summer12\_DR53X-PU\_S10\_START53\_V7C-v1/AODSIM \\
\hline
/VBF\_HToMM\_M-125\_TuneZ2star\_8TeV-powheg-pythia6/Summer12\_DR53X-PU\_S10\_START53\_V7C-v1/AODSIM \\
\hline
/WH\_HToMuMu\_M-125\_8TeV-powheg-herwigpp/Summer12\_DR53X-PU\_S10\_START53\_V19-v1/AODSIM \\
\hline
/ZH\_HToMuMu\_M-125\_8TeV-powheg-herwigpp/Summer12\_DR53X-PU\_S10\_START53\_V19-v1/AODSIM \\
\hline
\end{tabular}
\end{center}
\end{table}

% End Signal Samples
%%%%%%%%%%%%%%%%%%%%%%%%%%%%%%%%%%%%%%%%%%%%%%%%%%%%%%%%%%%%%%%%%%%%%%
% Begin Background Samples

\subsubsection{Simulated Background Samples}

The simulated background samples used in our analysis are listed in 
Table~\ref{tab:BGSamples8TeV}.

\begin{table}[h]
\begin{center}
\caption{\label{tab:BGSamples8TeV} Background MC samples at $\sqrt{s}=8$ TeV}
\begin{tabular}{|p{10cm}|c|c|c|c|} \hline
Sample & Original  & $\sigma$ [pb] & Equivalent \\
 &  \# Events &  &  Lumi [1/$fb$] \\
\hline
\hline
\scriptsize/DYJetsToLL\_M-50\_TuneZ2Star\_8TeV-madgraph-tarball/Summer12\_DR53X-PU\_S10\_START53\_V7A-v1/AODSIM &	     30086987 &	  3503.71 &   8.6  \\
\hline                                                                                                                                     
\scriptsize/TTJets\_MassiveBinDECAY\_TuneZ2star\_8TeV-madgraph-tauola/Summer12\_DR53X-PU\_S10\_START53\_V7A-v3/AODSIM &  6921652 &	  225.197 &  30.7  \\
\hline                                                                                                                                                       
\scriptsize/WW\_TuneZ2star\_8TeV\_pythia6\_tauola/Summer12\_DR53X-PU\_S10\_START53\_V7A-v1/AODSIM & 	              5218045 &	   54.838 &  95.2  \\
\hline                                                                                                                                     
\scriptsize/WZ\_TuneZ2star\_8TeV\_pythia6\_tauola/Summer12\_DR53X-PU\_S10\_START53\_V7A-v1/AODSIM &	             10000283 &	    33.21 & 301    \\
\hline                                                                                                                                     
\scriptsize/ZZ\_TuneZ2star\_8TeV\_pythia6\_tauola/Summer12\_DR53X-PU\_S10\_START53\_V7A-v1/AODSIM 	&  	              9799908 &    17.654 & 555	   \\
\hline
\end{tabular}
\end{center}
\end{table}

\subsection{Muon and Jet Object Definitions \label{MuonDef}}

In our analysis on simulated data, events are required to pass a simulated \texttt{HLT\_IsoMu24\_eta2p1} trigger.
In all of our analyses, events are also required to contain a pair of opposite-sign muons with $|\eta|<2.1$.
A muon must be matched to the trigger (within $\Delta R = \sqrt{(\Delta \phi)^2+(\Delta \eta)^2}< 0.2$), 
and have $p_T>25$\,GeV, while the other muon must have $p_T>15$\,GeV.
Both muons must pass the ``tight'' muon ID requirements~\cite{AN2012_459}.
as well as being isolated from other event activity.  We quantified isolation using
particle-flow (PF) relative isolation corrected for pileup using the ``$\delta \beta$''
procedure~\cite{AN2012_459}.  Each muon is required to have a PF relative isolation value
of less than 0.12.  Additionally, the 3D opening angle between the two muons must be
smaller than $\pi-0.02$ radians, and events must contain a well reconstructed primary
vertex.
In the case where there are more than two muons in an event, all pairs passing the above
requirements are considered, and will be referred to as separate events for the remainder
of this note.  

The simulated muon selection efficiency is corrected to match the data using 
the ``tag-and-probe'' method, using scale factors.  The muon momentum resolution
is also corrected to match the resolution of the MuScleFit-corrected data~\cite{AN2012_459}.

Jets are reconstructed from PF candidates clustered with the anti-$k_t$ algorithm with
a radius parameter of 0.5.  The $p_T$ and $|\eta|$ requirements of the jets are varied in
the below optimization.  A loose jet ID is applied~\cite{AN2012_459}, and a further
MVA-based selection is applied to reject jets formed from pileup particles~\cite{PUID}.

\subsection{VH Selection Criteria}

We began improving the $H \rightarrow \mu \mu$ analysis by creating and optimizing additional selection criteria in the VH category. 
Since all of categories in the baseline analysis were looking at $H \rightarrow \mu \mu$, we kept the same dimuon 
selection for VH (as specified in section \ref{MuonDef}). 
The baseline analysis selected jets with $p_{T}, \eta,$ and mass values expected in VBF, but in VH the jets originated from the vector boson. 
Therefore we altered the jet selection criteria in the VH category to specifically look for $V \rightarrow jj$.

In VBF, jets are predicted to have a high $p_{T}$ and a more forward position in the detector (i.e. high $|\eta|$). 
Comparatively, jets in VH are predicted to have a lower $p_{T}$ and a lower $|\eta|$. 
Therefore, we decided to investigate lowering our $p_{t}$ and $|\eta|$ criteria. After adding these basic selection criteria,
we began investigating more advanced selection criteria

One of our additional selection criteria that selects against events not containing a Z or W boson is our dijet Invariant mass cut. 
Since $V \rightarrow jj$, we expect that the dijet mass should be similar to the mass of the vector boson. 

The biggest background contributors in the VH category are Drell Yan ($DY$) and top quark decays ($t\overline{t}$), 
so our ultimate goal, with the jet selection, is to optimize the VH signal over these two background contributors.

This next cut is specifically aimed at selecting against $t\overline{t}$.
Since we are looking at associated production of a Higgs and a Vector boson, we expect that they will 
travel in opposite directions within the transverse plane of the detector. 
Therefore we decided to create a criterion based on the angle between $H$ and $V$. 
In order to accomplish this, we calculated the direction that $H$ was traveling based off of the dimuon pair 
and we calculated the direction that the $V$ was traveling based off of the dijet pair. 
The idea of using jet and dimuon $p_{T}$s lead us to another very similar cut that also selects against $t\overline{t}$ 

In the detector, the sum of the transverse momentum of all of the particles should be zero, 
but, due to measurement uncertainties and undetected particles, this is rarely the case. 
$t\overline{t}$ is more likely to have a larger amount of missing transverse momentum ($p_{T}^{miss}$) than the signal events. 
In the baseline analysis all events with $p_{T}^{miss} >$ 40 GeV/c are excluded.
After selecting against $t\overline{t}$ events, we decided we needed additional selection criteria against $DY$.

The Transverse Dimuon Momentum ($p_{T}^{\mu \mu}$) is a vector sum of the individual muon $p_{T}$s. 
We predicted that our signal events should have a longer tail than the $DY$ events.

\subsection{Statistical Methonds}

\subsubsection{Statistical Significance Estimate}

In order to determine how effective our selection was, we used a statistical significance estimate 
	\[\frac{S}{\sqrt{B}} \] 
(Where $S$ is the number of signal events and $B$ is the number of background events). 
This estimates the Z-score of the signal over the background allowing us to estimate how large our signal is compared to the background. 
In between selections, we took the ratio 
	\[\frac{\frac{S_{1}}{S_{0}}}{\sqrt{\frac{B_{1}}{B_{0}}}} \] 
(Where $S_{0}$ and $B_{0}$ are the signal and background events before the selection 
and $S_{1}$ and $B_{1}$ are the signal and background events after the selection). 
A ratio value larger than one signifies an improvement in the signal relative to the background.

\subsubsection{Expected Limits Definition}

Expected upper limits are computed using the asymptotic likelihood-ratio-based approach of 
Ref.~\cite{stats}.  This is done with the CMS Higgs group ``combine'' tool using binned
fits of parameterized signal and background shapes to the dimuon invariant mass 
($m_{\mu\mu}$) data.  The signal shape is parameterized
as the sum of two Gaussians, with parameters fixed to values fit to signal MC.
The background shape is a Breit--Wigner plus $1/m_{\mu\mu}^2$ term, both multiplied
by an exponential.  The Breit-Wigner mass and width are pre-fit to Z-peak data and fixed.
The expected limit is computed
using background shapes fit to 19.7\,fb$^{-1}$ of 8\,TeV data in each category,
and assuming no Higgs signal (they are background-only expected limits).  The upper
limits are presented in terms of the $\mathrm{CL_s}$ criterion~\cite{cls}, and are computed for 
a 95\% confidence level.  These expected limits include no systematic uncertainties.

\section{Results}

\subsection{VH Analysis Optimization 1 (for 8 TeV Simulated data)}

\subsubsection{Selection Optimization}

We began our optimization of the VH category by applying the VH selection criteria in an analysis on 8 TeV simulated data. 

In our selection optimization, all iterations of the analysis included a VH category with the dimuon selection as a precondition. 
From here, we optimized the basic jet selection criteria on the leading and sub-leading jets. 
In our optimization of the jet $p_{T}$ we found that selection of $p_{T}>$ 25 GeV/c was the best. 
This selection has $\frac{S}{\sqrt{B}}$ = 1.054 when compared to an initial jet $p_{T}$ selection of $p_{T}>$ 20 GeV/c. 
However, due to an expected increase in detector uncertainties as the collision energy increases to 13 TeV, 
we decided that the leading and sub-leading jet $p_{T}$ selection had to be no less than $p_{T}>$ 30 GeV/c. 
We found that this selection, which has $\frac{S}{\sqrt{B}}$ = 0.92 when compared to the initial jet $p_{T}$ selection of $p_{T}>$ 20 GeV/c, 
was better than any of the higher selections. 

After applying a Jet $p_{T}$ cut of 30 GeV/c to the $VH$ category, we began investigating Jet $|\eta|$ using the jet $p_{T}$ selection as a precondition.  
We found that a selection of $|\eta|<$ 2.7 for was the best, with $\frac{S}{\sqrt{B}}$ = 1.044 when compared to an initial selection of $|\eta|<$ 4.7. 
Then we used the Dimuon selection, $|\eta|$ selection, and $p_{T}$ selection as preconditions for optimizing the Dijet mass ($M_{jj}$) cut. 
Figure \ref{fig:Mjj} shows the Dijet mass and $|\eta|$ distributions that we applied the selection criteria to.
\begin{figure}[!hbtp]
\begin{center}
    \includegraphics[width=0.49\textwidth]{images/Hist_jetMass.png}
    \includegraphics[width=0.49\textwidth]{images/Hist_jetEta.png}
    \caption{ \label{fig:Mjj}
         a) This plot shows the 2 Jet invariant mass after the preconditions have been applied. b) This plot shows the $|\eta|$
	 values after the Dimuon selection and Jet $p_{T}$ selection. The orange and light green events are the signal events and the other colors
	 are background.
      }
\end{center}
\end{figure}
For this criteria, we found that a mass selection of 60 GeV/c$^{2}< M_{jj}<$ 110 GeV/c$^{2}$ was the best
because it encompassed both the W peak and the Z peak. 
This cut has $\frac{S}{\sqrt{B}}$ = 1.24. For the rest of the selection criteria optimization we used only the following preconditions: 
the Dimuon selection, the Jet $p_{T}$ cut, the Jet $\eta$ cut, and the Dijet mass cut.

After optimizing our basic selection criteria and determining our preconditions, 
we optimized the selections on the transverse Angle Between the Higgs and vector boson ($\Delta\phi_{\mu\mu jj}$), 
missing transverse Momentum and dimuon transverse momentum. 
We found a selection of $\cos(\Delta\phi_{\mu\mu jj})<$ -0.95 to be the best (Figure \ref{fig:cosPhi}).
\begin{figure}[!hbtp]
\begin{center}
    \includegraphics[width=0.49\textwidth]{images/Hist_MuMu2JetPhiBeforeCuts.png} %Redo this plot!!!!
    \includegraphics[width=0.49\textwidth]{images/Hist_MuMu2JetPhiZoomB4Cut.png}
    \caption{ \label{fig:cosPhi}
         a) This plot shows the $\cos(\Delta\phi_{\mu\mu jj})$ value after the preconditions have been applied. b) This plot is a close-up of plot a.
	 The orange and light green events are the signal events and the rest others are background.
      }
\end{center}
\end{figure} 
This cut has $\frac{S}{\sqrt{B}}$ = 1.18. For the missing transverse momentum we found a selection of 
$p_{T}^{miss} <$ 40 GeV/c to be the best (Figure \ref{fig:ptmiss}a).
This cut has $\frac{S}{\sqrt{B}}$ = 1.12. Finally, we found a selection of dimuon $p_{T} >$ 110 GeV/c to be the best 
(Figure \ref{fig:ptmiss}b). This cut has $\frac{S}{\sqrt{B}}$ = 1.20. 
When we combine all of the selection criteria, $\frac{S}{\sqrt{B}} $ = 1.82.
\begin{figure}[!hbtp]
\begin{center}
    \includegraphics[width=0.49\textwidth]{images/Hist_PtMiss.png}
    \includegraphics[width=0.49\textwidth]{images/Hist_DiMuonPt.png}
    \caption{ \label{fig:ptmiss}
         a) This plot shows the $p^{miss}_{T}$ values after the preconditions have been applied. b) This plot shows the dimuon $p_{T}$ values 
	 after the preconditions have been applied. The orange and light green events are the signal events and the rest others are background.
      }
\end{center}
\end{figure} 

%<Table with S/sqrtB ratio values>

\subsubsection{Choosing Correct Dijet Pair}

In the VBF category, selection took place on the two highest $p_{T}$ jets. 
Similarly, we started off using the two highest $p_{T}$ jets for the VH category, 
but we soon discovered that some of the dijet combinations had unusually extreme values. 
Therefore, we decided to alter our selection criteria to see if we could pick up the correct jets. 
Instead of taking the highest $p_{T}$ jets, we selected for the highest dijet $p_{T}$ pair. This is shown 
in figure \ref{fig:dijetSel}. This improved our peak, but we still seem to be missing jets.
\begin{figure}[!hbtp]
\begin{center}
    \includegraphics[width=0.49\textwidth]{images/Hist_Reco2PlotsPt30Eta2.png}
    \caption{ \label{fig:dijetSel}
         This shows the dijet pair selection in $WH$ for the event pair with a mass closest to 80 (a flawed selection that 
	 shows the best possible values; represented by the green line), the event pair with the highest $p_{T}$ (black line), 
	 and the event pair created from the two highest $p_{T}$ jets (red line).
      }
\end{center}
\end{figure}

\subsection{VH Analysis Optimization 2 (for 8 TeV data)}

After obtaining optimum values for the selection criteria in the VH category, we applied these categories to the full 8 TeV analysis on real data. 
We then re-optimized the VH category based off of expected limits. 

The baseline analysis on 8 TeV data includes a 2-jet category. This category requires (after dimuon selection) 
that the leading jet has $p_{T} >$ 40 GeV/c, the sub-leading jet has $p_{T} >$ 30 GeV/c, and that the event has $p_{T}^{miss} <$ 40 GeV/c. 
In order for the new category to run
on the full analysis, without significant alterations to the analysis, the jet $p_{T}$ values cannot change. Using only expected limit comparisons for 
optimization, we found that a selection of $p_{T}>$ 110 GeV/c for dimuon $p_{T}$, 
a selection of 60 GeV/c$^{2}< M_{jj}<$ 110 GeV/c$^{2}$ for dijet mass, and a selection of $|\eta|<$ 2.7 for leading and sub-leading 
jet $|\eta|$ to have the best expected limit value at 92.5781, when combined with the baseline 2-jet cuts. 
The expected limit values are shown in table \ref{tab:ExpectedLimits}.

\begin{table}[!hbtp]
  \begin{center}
    \caption{ \label{tab:ExpectedLimits}
        This Table is an optimization flow table of the $VH$ Category with preconditions of leading jet $p_{T} >$ 40
	GeV/c, sub-leading jet $p_{T} >$ 30 GeV/c, and $p_{T}^{miss} <$ 40 GeV/c.
    }
    \begin{tabular}{lcc} \hline \hline
         Selection Criteria & Expected limit \\ \hline
         60 GeV/c$^{2} < M_{jj} <$ 110 GeV/c$^{2}$ & 96.0938  \\
         + Dimuon $p_{T} >$ 110 GeV/c & 93.0469 \\
	 + Leading Jet $|\eta| <$ 2.7 & 93.0469 \\
	 + Sub-Leading Jet $|\eta| <$ 2.7 & 92.5781 \\
     \hline \hline
    \end{tabular}
  \end{center}
\end{table}

\section{Discussion}



\begin{thebibliography}{9}

\bibitem{HiggsDisc} 
  S.~Chatrchyan {\it et al.}  [CMS Collaboration],
  ``Observation of a new boson at a mass of 125 GeV with the CMS experiment at the LHC,''
  Phys.\ Lett.\ B {\bf 716}, 30 (2012)
  [arXiv:1207.7235 [hep-ex]].
  %%CITATION = ARXIV:1207.7235;%%
  %3905 citations counted in INSPIRE as of 18 Feb 2015

\bibitem{hmumuPap} 
  CMS Collaboration [CMS Collaboration],
  ``Search for the standard model Higgs boson in the dimuon decay channel in pp collisions at sqrt(s)= 7 and 8 TeV,''
  CMS-PAS-HIG-13-007.
  %%CITATION = CMS-PAS-HIG-13-007;%%
  %11 citations counted in INSPIRE as of 03 Sep 2014

\bibitem{snow} 
  [CMS Collaboration],
  ``Projected Performance of an Upgraded CMS Detector at the LHC and HL-LHC: Contribution to the Snowmass Process,''
  arXiv:1307.7135.
  %%CITATION = ARXIV:1307.7135;%%
  %62 citations counted in INSPIRE as of 07 Oct 2014


\bibitem{AN2012_459}
  Acosta, D., et. al.,
  ``Search for standard model Higgs boson production in the $\mu^+\mu^-$ final state with the CMS experiment in pp collisions at $\sqrt{s}=7$ and 8\, TeV'',
  CMS Analysis Note AN-12459,
  2014.

\bibitem{powheg1} 
  S.~Alioli, P.~Nason, C.~Oleari and E.~Re,
  ``A general framework for implementing NLO calculations in shower Monte Carlo programs: the POWHEG BOX,''
  JHEP {\bf 1006}, 043 (2010)
  [arXiv:1002.2581 [hep-ph]].
  %%CITATION = ARXIV:1002.2581;%%

\bibitem{powheg2} 
  S.~Alioli, P.~Nason, C.~Oleari and E.~Re,
  ``NLO Higgs boson production via gluon fusion matched with shower in POWHEG,''
  JHEP {\bf 0904}, 002 (2009)
  [arXiv:0812.0578 [hep-ph]].
  %%CITATION = ARXIV:0812.0578;%%

\bibitem{powheg3} 
  P.~Nason and C.~Oleari,
  ``NLO Higgs boson production via vector-boson fusion matched with shower in POWHEG,''
  JHEP {\bf 1002}, 037 (2010)
  [arXiv:0911.5299 [hep-ph]].
  %%CITATION = ARXIV:0911.5299;%%

\bibitem{pythia} 
  T.~Sjostrand, S.~Mrenna and P.~Z.~Skands,
  ``PYTHIA 6.4 Physics and Manual,''
  JHEP {\bf 0605}, 026 (2006)
  [hep-ph/0603175].

\bibitem{herwigpp} 
  M.~Bahr, S.~Gieseke, M.~A.~Gigg, D.~Grellscheid, K.~Hamilton, O.~Latunde-Dada, S.~Platzer and P.~Richardson {\it et al.},
  ``Herwig++ Physics and Manual,''
  Eur.\ Phys.\ J.\ C {\bf 58}, 639 (2008)
  [arXiv:0803.0883 [hep-ph]].

\bibitem{PUID}
  CMS Collaboration, ``Pileup Jet Identification'', 
  CMS Physics Analysis Summary CMS-PAS-JME-13-005, 
  2013.

\bibitem{stats}
  ATLAS and CMS Collaborations, and LHC Higgs Combination Group, ``Procedure for
  the LHC Higgs boson search combination in summer 2011'', 
  CMS-NOTE-2011/005 ATL-PHYS-PUB-2011-011, 2011.

\bibitem{cls}
  A. L. Read, ``Presentation of search results: the CLs technique'', J. Phys. G 28 (2002) 2693,
  doi:10.1088/0954-3899/28/10/313.

\end{thebibliography}

\end{document}
