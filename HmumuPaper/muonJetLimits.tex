\documentclass[12pt]{article}
 
\begin{document}

\section{Muon and Jet Object Definitions}
 
Events are required to pass a simulated \texttt{HLT\_IsoMu24\_eta2p1} trigger.
They are also required to contain a pair of opposite-sign muons with $|\eta|<2.1$.
A muon must be matched to the trigger (within $\Delta R = \sqrt{(\Delta \phi)^2+(\Delta \eta)^2}< 0.2$), 
and have $p_T>25$\,GeV, while the other muon must have $p_T>15$\,GeV.
Both muons must pass the ``tight'' muon ID requirements~\cite{AN2012_459}.
as well as being isolated from other event activity.  Isolation is quantified using
particle-flow (PF) relative isolation corrected for pileup using the ``$\delta \beta$''
procedure~\cite{AN2012_459}.  Each muon is required to have a PF relative isolation value
of less than 0.12.  Additionally, the 3D opening angle between the two muons must be
smaller than $\pi-0.02$ radians, and events must contain a well reconstructed primary
vertex.
In the case where there are more than two muons in an event, all pairs passing the above
requirements are considered, and will be referred to as separate events for the remainder
of this note.  

The simulated muon selection efficiency is corrected to match the data using 
the ``tag-and-probe'' method, using scale factors.  The muon momentum resolution
is also corrected to match the resolution of the MuScleFit-corrected data~\cite{AN2012_459}.

Jets are reconstructed from PF candidates clustered with the anti-$k_t$ algorithm with
a radius parameter of 0.5.  The $p_T$ and $|\eta|$ requirements of the jets are varied in
the below optimization.  A loose jet ID is applied~\cite{AN2012_459}, and a further
MVA-based selection is applied to reject jets formed from pileup particles~\cite{PUID}.

\section{Expected Upper Limit Computation Procedure}

Expected upper limits are computed using the asymptotic likelihood-ratio-based approach of 
Ref.~\cite{stats}.  This is done with the CMS Higgs group ``combine'' tool using binned
fits of parameterized signal and background shapes to the dimuon invariant mass 
($m_{\mu\mu}$) data.  The signal shape is parameterized
as the sum of two Gaussians, with parameters fixed to values fit to signal MC.
The background shape is a Breit--Wigner plus $1/m_{\mu\mu}^2$ term, both multiplied
by an exponential.  The Breit-Wigner mass and width are pre-fit to Z-peak data and fixed.
The expected limit is computed
using background shapes fit to 19.7\,fb$^{-1}$ of 8\,TeV data in each category,
and assuming no Higgs signal (they are background-only expected limits).  The upper
limits are presented in terms of the $\mathrm{CL_s}$ criterion~\cite{cls}, and are computed for 
a 95\% confidence level.  These expected limits include no systematic uncertainties.

\begin{thebibliography}{9}

\bibitem{AN2012_459}
  Acosta, D., et. al.,
  ``Search for standard model Higgs boson production in the $\mu^+\mu^-$ final state with the CMS experiment in pp collisions at $\sqrt{s}=7$ and 8\, TeV'',
  CMS Analysis Note AN-12459,
  2014.

\bibitem{PUID}
  CMS Collaboration, ``Pileup Jet Identification'', 
  CMS Physics Analysis Summary CMS-PAS-JME-13-005, 
  2013.

\bibitem{stats}
  ATLAS and CMS Collaborations, and LHC Higgs Combination Group, ``Procedure for
  the LHC Higgs boson search combination in summer 2011'', 
  CMS-NOTE-2011/005 ATL-PHYS-PUB-2011-011, 2011.

\bibitem{cls}
  A. L. Read, ``Presentation of search results: the CLs technique'', J. Phys. G 28 (2002) 2693,
  doi:10.1088/0954-3899/28/10/313.

\end{thebibliography}
 
\end{document}
